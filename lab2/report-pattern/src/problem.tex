\CWHeader{Лабораторная работа \textnumero 2}

\CWProblem{
Необходимо создать программную библиотеку, реализующую указанную структуру данных, на основе которой разработать программу-словарь. В словаре каждому ключу, представляющему из себя регистронезависимую последовательность букв английского алфавита длиной не более 256 символов, поставлен в соответствие некоторый номер, от 0 до 264 - 1. Разным словам может быть поставлен в соответствие один и тот же номер.

Программа должна обрабатывать строки входного файла до его окончания. Каждая строка может иметь следующий формат:
    
{\bfseries + word 34} — добавить слово «word» с номером 34 в словарь. Программа должна вывести строку «OK», если операция прошла успешно, «Exist», если слово уже находится в словаре.
    
{\bfseries - word} — удалить слово «word» из словаря. Программа должна вывести «OK», если слово существовало и было удалено, «NoSuchWord», если слово в словаре не было найдено.
    
{\bfseries word} — найти в словаре слово «word». Программа должна вывести «OK: 34», если слово было найдено; число, которое следует за «OK:» — номер, присвоенный слову при добавлении. В случае, если слово в словаре не было обнаружено, нужно вывести строку «NoSuchWord».
    
{\bfseries ! Save /path/to/file} — сохранить словарь в бинарном компактном представлении на диск в файл, указанный парамером команды. В случае успеха, программа должна вывести «OK», в случае неудачи выполнения операции, программа должна вывести описание ошибки (см. ниже).
    
{\bfseries ! Load /path/to/file} — загрузить словарь из файла. Предполагается, что файл был ранее подготовлен при помощи команды Save. В случае успеха, программа должна вывести строку «OK», а загруженный словарь должен заменить текущий (с которым происходит работа); в случае неуспеха, должна быть выведена диагностика, а рабочий словарь должен остаться без изменений. Кроме системных ошибок, программа должна корректно обрабатывать случаи несовпадения формата указанного файла и представления данных словаря во внешнем файле.
    
Для всех операций, в случае возникновения системной ошибки (нехватка памяти, отсутсвие прав записи и т.п.), программа должна вывести строку, начинающуюся с «ERROR:» и описывающую на английском языке возникшую ошибку.
\\{\bfseries Вариант структуры данных:} {  \normalfont\ttfamily Красно-чёрное дерево.}
}

\pagebreak
