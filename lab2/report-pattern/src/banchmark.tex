\section{Тест производительности}
Время на ввод данных не учитывается, замеряется только время, затраченное на операции классов std::map и TRBTree. Количество комманд для каждого файла равно десять в степени номер теста. Например, $02.t$ содержит сто пар, а $06.t$ миллион.

\begin{alltt}
nezhov@LAPTOP-VDTFKP8R:/mnt/c/Users/nikit/Diskran/lab2$ ./benchmark < tests/03.t
==============START============
INSERT std::map time: 4 ms
INSERT rb tree time: 444 ms
===============================
DELETE std::map time: 5 ms
DELETE rb tree time: 65 ms
===============================
SEARCH std::map time: 0 ms
SEARCH rb tree time: 8 ms
==============END==============
nezhov@LAPTOP-VDTFKP8R:/mnt/c/Users/nikit/Diskran/lab2$ ./benchmark < tests/05.t
==============START============
INSERT std::map time: 412 ms
INSERT rb tree time: 33463 ms
===============================
DELETE std::map time: 471 ms
DELETE rb tree time: 11843 ms
===============================
SEARCH std::map time: 78 ms
SEARCH rb tree time: 8259 ms
==============END==============
    
\end{alltt}

Глядя на результаты, видно, что $std::map$ выигрывает на всех тестах. Видимо, создатели стандартной библиотеки потратили много времени на оптимизацию работы $std::map$.
\pagebreak

