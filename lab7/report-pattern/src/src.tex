\section{Описание}
Требуется описать метод динамического программирования, который позволит решить поставленную задачу. Итак, нам нужно
уменьшить число $n$ до 1 имея три операции ($-1 /2 /3$). Для того, чтобы быстро найти ответ построим массив всех ответов $ans$
(массив состоит из элементов, которые содержат в себе мин. стоимость cost и следующий шаг operation), где по индексу $i$ будет 
храниться минимальная стоимость превращения $i$ в 1, а также одна из 3-х операций, которая является
следующим шагом для уменьшения числа до 1. Находить ответы будем от меньшего к большему, т.е. мы начнём с 1 и будем идти до $10^7$.
Наименьшая стоимость на текущем шаге складывается из числа $i + min(ans[i-1].count, min(ans[i/3].count, ans[i/2].count))$. Естественно, 
чтобы проверить $ans[i/2].count$ и $ans[i/3].count$ мы сначала проверяем $i\%2==0$ и $i\%3==0$. Найдя минимум, прибавляем его к нашему
текущему числу (это и будет минимальная стоимость для текущего числа), а операцию приравниваем в соответствии с минимумом (если минимальный
$i-1$ то -1, если $i/2$ то /2, если $i/3$ то /3). После того, как мы подсчитали $ans[i]$ делаем $i++$, и начинаем алгоритм, описанный выше,
сначала. Оценка построения массива ответов $O(3n)$, но т.к. константы не учитываются, то сложность будет $O(n)$. А время получения ответа для
конкретного числа также не может быть больше $O(n)$ (нам приходиться идти по всему массиву до 1, т.к. для хранения всех операций для каждого числа
потребуется слишком много памяти + активное копирование строк замедляет программу).
\pagebreak

\begin{longtable}{|p{7.5cm}|p{7.5cm}|}
    \hline
    \rowcolor{lightgray}
    \multicolumn{2}{|c|} {TNumber}\\
    \hline
    TNumber()&Конструктор для числа.\\
    \hline
    long long const&Минимальная стоимость преобразования числа в 1.\\
    \hline
    std::string *operation&Указатель на строку, содержащую в себе операцию, которая ведёт к минимальному пути.\\
    \hline
    bool operator <(TNumber \&lhs, TNumber \&rhs)&Оператор сравнения TNumber, который сравнивает поля cost.\\
    \hline
    std::ostream \&operator << (std::ostream \&of, TNumber \&number)&Оператор вывода для TNumber (выводит операцию минимального пути).\\
    \hline
\end{longtable}

\begin{longtable}{|p{7.5cm}|p{7.5cm}|}
    \hline
    \rowcolor{lightgray}
    \multicolumn{2}{|c|} {TDynVector}\\
    \hline
    TDynVector() = default&Стандартный конструктор для массива ответов.\\
    \hline
    TDynVector(long long size)&Конструктор с указанием максимального числа $n$.\\
    \hline
    void Print(long long n)&Распечатать решение минимальное решение для числа $n$.\\
    \hline
    void PrintHelper(long long n)&Вспомагательная рекурсивная функция для печати оставшейся части решения.\\
    \hline
    TNumber\& operator[](const long long index)&Оператор для получения элемента по индексу.\\
    \hline
    ~TDynVector()&Деструктор для массива ответов.\\
    \hline
    void CountCost()&Функция подсчёта решений для всех чисел $2 \leq i \geq 10^7$.\\
    \hline
    lonh lonh Size&Размер массива ответов.\\
    \hline
    TNumber *Data&Указатель на массив, где хранятся решения.\\
    \hline
    std::string empty&Пустая строка.\\
    \hline
    std::string fType = "-1 "\ &Строка действия, указатель на которую будет хранится в TNumber.operation (также для "/2 " и "/3 ").\\
    \hline
\end{longtable}


\pagebreak
\section{Исходный код}
\begin{lstlisting}[language=C++]
//
// struct.hpp
//
#include <string>
#include <iostream>
#include <vector>

struct TNumber{
    long long cost;
    std::string *operation;
    TNumber(){
        cost = 0;
    }
};

bool operator <(TNumber &lhs, TNumber &rhs){
    return lhs.cost < rhs.cost;
}

std::ostream &operator << (std::ostream &of, TNumber &number){
    of << *number.operation;
    return of;
}

class TDynVector{
private:
    std::string empty;
    std::string fType = "-1 ";
    std::string sType = "/2 ";
    std::string tType = "/3 ";
    TNumber *Data;
    long long Size = 0;
    void CountCost(){
        Data[1].operation = &empty;
        long long currIndex = 2;
        std::string *operation = &empty;
        while(currIndex != Size + 1){
            long long minCost = -1;
            if(currIndex % 3 == 0){
                minCost = Data[currIndex/3].cost;
                operation = &tType;
            }
            if(currIndex % 2 == 0 && (Data[currIndex/2].cost < minCost || minCost == -1)){
                minCost = Data[currIndex / 2].cost;
                operation = &sType;
            }
            if (currIndex - 1 > 0 && (Data[currIndex-1].cost < minCost || minCost == -1)){
                minCost = Data[currIndex - 1].cost;
                operation = &fType;
            }
            Data[currIndex].cost = minCost + currIndex;
            Data[currIndex].operation = operation;
            ++currIndex;
        }
    }
public:
    TDynVector() = default;
    TDynVector(long long size){
        Data = new TNumber[size + 1];
        this->Size = size;
        this->CountCost();
    }
    void Print(long long n){
        std::cout << this->Data[n].cost << "\n";
        PrintHelper(n);
    }
    void PrintHelper(long long n){
        std::cout << this->Data[n];
        if(n != 1) {
            if (this->Data[n].operation == &tType) {
                PrintHelper(n / 3);
            } else if (this->Data[n].operation == &sType) {
                PrintHelper(n / 2);
            } else if (this->Data[n].operation == &fType) {
                PrintHelper(n - 1);
            }
        }
    }
    TNumber& operator[](const long long index){
        return Data[index];
    }
    ~TDynVector(){
        delete [] Data;
    }
};
\end{lstlisting}

\begin{lstlisting}[language=C++]
//
// main.cpp
//
#include "struct.hpp"
int main(){
    std::cin.tie(nullptr);
    std::cout.tie(nullptr);
    std::ios_base::sync_with_stdio(false);
    long long n;
    TDynVector dynVector(10000000);
    if(std::cin >> n){
        dynVector.Print(n);
    }
    std::cout << "\n";
    return 0;
}

\end{lstlisting}
\pagebreak

\section{Консоль}
\begin{alltt}
{\color{blue} nezhov@killswitch:~/CLionProjects/Diskran/lab7$} make
g++ -std=c++14 -O3 -Wextra -Wall -Werror -Wno-sign-compare -Wno-unused-result -pedantic -o solution main.cpp
{\color{blue} nezhov@killswitch:~/CLionProjects/Diskran/lab7$} ./solution
82
202
-1 /3 /3 /3 /3 
\end{alltt}
\pagebreak

