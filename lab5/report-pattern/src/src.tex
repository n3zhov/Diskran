\section{Описание}
Требуется написать реализацию алгоритма Укконнена для построения суффиксного дерева и при помощи него решить поставленную задачу.
Мы будем строить явное суффиксное дерево, терминирующим символом будет $\$$.
Опишем т.н. {\bfseries правила продолжения суффиксов:}
\\Укажем точные правила для \textit {продолжения суффикса}. Пусть $S[j..i] = \beta$ - суффикс $S[1..i]$. В продолжении $j$, когда
алгоритм находит конец $\beta$ в текущем дереве, он продолжает $\beta$, чтобы обеспечить присутствие суффикса $\beta S (i+1)$ в
дереве. Алгоритм действует по одному из следующих трёх правил:
\\{\bfseries Правило 1.} В текущем дереве путь $\beta$ кончается в листе. Это значит, что путь от корня с меткой $\beta$ доходит
до конца некоторой "листовой"\ дуги (дуги, входящие в лист). При изменеии дерева нужно добавить к концу метки этой листовой дуги $S(i+1)$.
\\{\bfseries Правило 2.} Ни один путь из конца строки $\beta$ не начинается символом $S(i+1)$, но по крайне мере один начинающийся оттуда
путь имеется.
В этом случае должна быть создана новая листовая дуга, начинающаяся в конце $\beta$ и помеченная символом $S(i+1)$. При этом, если $\beta$
кончается внутри дуги, должна быть создана новая вершина. Листу в конце новой листовой дуги сопоставляется номер $j$.
\\{\bfseries Правило 3.} Некоторый путь из конца строки $\beta$ начинается символом $S(i+1)$. В этом случае строка $\beta S(i+1)$ уже имеется
в текущем дереве, так что ничего не надо делать.
\\Для ускорения алгоритма используем суффиксные связи. Пусть $x\alpha$ обозначает произвольную строку, где $x$ - ее первый символ, а $\alpha$
оставшаяся подстрока (возможно, пустая). Если для внутренней вершины $v$ с путевой меткой $x\alpha$ существует другая вершина $s(v)$ с путевой
отметкой $\alpha$, то указатель из $v$ в $s(v)$ называется \textit{суффиксной связью}. Суффиксную связь будем обозначать парой $(v, s(v))$.
Если строка $\alpha$ пуста, связь идёт в корневую вершину. Но это лишь уменьшает кол-во передвижений от корня в каждом продолжении.
\\Для оптимизации алгоритма используем 3 приёма
\\{\bfseries Приём 1.}
\\"Перескок". Суть этого приёма заключается в том, что если мы хотим найти место для вставки вхождения строки в дерево, то мы не идём посимвольно,
если длина строки $\gamma$ больше длины текущей дуги. В таком случае мы её просто пропускаем, а идти начинаем от $|\gamma|-$ длина дуги.
\\{\bfseries Приём 2.}
\\Заканчивать каждую фазу $i+1$ после первого же использования правила 3. Если это случится в продолжении $j$, то уже не требуется явно находить
концов строк $S[k..i]$ с $k > j$. Также стоит учесть, что дуги мы "сжимаем"\ до дуговых меток, т.е. это у нас не массив суффиксов, а пара чисел $(p, e)$, 
где $e$ - "текущий конец"\
\\{\bfseries Приём 3.}
\\Если дуга стала листовой - она таковой и останется. Из этого следует, что $e$ для листов можно ввести как глобальную переменную, и работать с 
ней за константу.
\pagebreak

\begin{longtable}{|p{7.5cm}|p{7.5cm}|}
    \hline
    \rowcolor{lightgray}
    \multicolumn{2}{|c|} {TNode}\\
    \hline
    TNode(TNode *link, int start, int *end)&конструктор дуги без индекса, используется при создании новых внутренних вершин\\
    \hline
    \hline
    TNode(TNode *link, int start, int *end, int int)&Конструктор с индексом используется при создании лисьтев\\
    \hline
    map<char, TNode *> Children&Массив "детей".\\
    \hline
    TNode* SuffixLink&Суффиксная ссылка.\\
    \hline
    int Start&Индекс первого символа вершины в тексте.\\
    \hline
    \hline
    int *End&Индекс последнего символа вершины в тексте\\
    \hline
    \hline
    int SuffixIndex&Индекс суффикса.\\
    \hline
\end{longtable}

\begin{longtable}{|p{7.5cm}|p{7.5cm}|}
    \hline
    \rowcolor{lightgray}
    \multicolumn{2}{|c|} {TSuffixTree}\\
    \hline
    TSuffixTree(string \&text)&Конструктор дерева из строки.\\
    \hline
    \hline
    void BuildSuffixTree()&Функция построения суффиксного дерева.\\
    \hline
    std::string Linearization(int n)&Функция линеаризации среза циклической строки.\\
    \hline
    ~TSuffixTree()&Деструктор\\
    \hline
    void LinHelp(TNode *curr, int n, std::string \&res)&Вспомагательная рекурсивная функция для вычисления линеаризованной строки.\\
    \hline
    \hline
    void ExtendSuffixTree(int pos)&Расширение дерева.\\
    \hline
    \hline
    void DeleteSuffixTree(TNode *TNode)&Вспомагательная функция для удаления суффиксного дерева.\\
    \hline
    \hline
    int EdgeLength(TNode *TNode)&Вычисляет длину дуги.\\
    \hline
    \hline
    TNode *Root&Указатель на корневой узел дерева.\\
    \hline
    \hline
    TNode *LastCreatedInternalNode&Последняя созданная внутренняя вершина.\\
    \hline
    \hline
    string Text&Текст, на основе которого было построено суфф. дерево.\\
    \hline
    \hline
    TNode *ActiveNode&То, откуда начнется расширение на следующей фазе.\\
    \hline
    \hline
    int ActiveEdge&Индекс символа, который задает движение из текущей ноды\\
    \hline
    \hline
    int ActiveLength&на сколько символов идём в направлении activeEdge.\\
    \hline
    \hline
    int RemainingSuffixCount&Сколько суффиксов осталось создать.\\
    \hline
    \hline
    int LeafEnd&Глобальная переменная, "конец"\ листовой дуги.\\
    \hline
\end{longtable}
Для того, чтобы линеаризовать срез циклической строки, надо понять, сколько срезов нам понадобится для нахождения минимального
лексигографического среза. Допустим, что самый "маленький" символ находится в конце среза, и он в нём единственный. Это будет худший случай,
для которого потребуется $n+n-1$ символов. Возьмём два среза для простоты, т.е. $2n$. Как найти минимальный срез при помощи суфф. дерева: от каждой вершины
у нас идут дуги, нужно просто каждый раз выбирать минимальную дугу и доходить до конца суффикса. Если суффикс, например, размера меньше n, то надо
вернуться на шаг назад и взять следующую по возрастанию дугу и т.д., пока мы не найдём нужный нам минимальный срез. Не во всех случаях найденный суффикс
будет являться минимальным срезом строки, т.к. он может быть размера больше $n$, поэтому от первого такого найденного суффикса мы возвращаем первые $n$ символов,
это и будет наш ответ.

\pagebreak
\section{Исходный код}
\begin{lstlisting}[language=C++]
//
// suffTree.cpp
//

#pragma once

#include <iostream>
#include <string>
#include <vector>
#include <algorithm>
#include <map>

#define TERMINATION_SYMBOL '$'

using namespace std;

class TSuffixTree;

class TNode
{
private:
    map<char, TNode *> Children; 
    TNode *SuffixLink;
    int Start;
    int *End;
    int SuffixIndex;
public:
    friend TSuffixTree;


    TNode(TNode *link, int start, int *end) : TNode(link, start, end, -1){
    }


    TNode(TNode *link, int start, int *end, int ind) :
    SuffixLink(link),Start(start),End(end),SuffixIndex(ind){
    }
};

class TSuffixTree
{
private:
    void LinHelp(TNode *curr, int n, std::string &res){
        if (!curr)
            return;
        for (auto it : curr->Children) {
            LinHelp(it.second, n, res);
            if(res != "") {
                return;
            }
        }
        if (curr->SuffixIndex != -1 && n <= *curr->End - curr->SuffixIndex + 1
            && Text[curr->SuffixIndex + n - 1] != TERMINATION_SYMBOL) {
            res = Text.substr(curr->SuffixIndex, n);
        }
    }
    void ExtendSuffixTree(int pos){

        LastCreatedInternalNode = nullptr;
        LeafEnd++;
        RemainingSuffixCount++;
        while (RemainingSuffixCount > 0){
            if (ActiveLength == 0) {
                ActiveEdge = pos;
            }

            auto find = ActiveNode->Children.find(Text[ActiveEdge]);

            if (find == ActiveNode->Children.end()){
                ActiveNode->Children.insert(make_pair(Text[ActiveEdge],
                                                      new TNode(Root, pos, &LeafEnd, pos - RemainingSuffixCount + 1)));

                if (LastCreatedInternalNode != nullptr){
                    LastCreatedInternalNode->SuffixLink = ActiveNode;
                    LastCreatedInternalNode = nullptr;
                }
            }
            else{

                TNode *next = find->second;
                int edge_length = EdgeLength(next);

                if (ActiveLength >= edge_length){
                    ActiveEdge += edge_length;
                    ActiveLength -= edge_length;
                    ActiveNode = next;
                    continue;
                }

                if (Text[next->Start + ActiveLength] == Text[pos]){
                    if (LastCreatedInternalNode != nullptr && ActiveNode != Root) {
                        LastCreatedInternalNode->SuffixLink = ActiveNode;
                    }
                    ActiveLength++;
                    break;
                }

                TNode *split = new TNode(Root, next->Start, new int(next->Start + ActiveLength - 1));
                ActiveNode->Children[Text[ActiveEdge]] = split;
                next->Start += ActiveLength;
                split->Children.insert(make_pair(Text[pos], new TNode(Root, pos, &LeafEnd, pos - RemainingSuffixCount + 1)));
                split->Children.insert(make_pair(Text[next->Start], next));
                if (LastCreatedInternalNode != nullptr) {
                    LastCreatedInternalNode->SuffixLink = split;
                }
                LastCreatedInternalNode = split;
            }

            RemainingSuffixCount--;

            if (ActiveNode == Root && ActiveLength > 0){
                ActiveLength--;
                ActiveEdge++;
            }
            else if (ActiveNode != Root) {
                ActiveNode = ActiveNode->SuffixLink;
            }
        }
    }
    void DeleteSuffixTree(TNode *TNode){
        for (auto it : TNode->Children) {
            DeleteSuffixTree(it.second);
        }
        if (TNode->SuffixIndex == -1)
            delete TNode->End;
        delete TNode;
    }
    int EdgeLength(TNode *TNode){
        return *TNode->End - TNode->Start + 1;
    }


    TNode *Root = new TNode(nullptr, -1, new int(-1));
    TNode *LastCreatedInternalNode = nullptr;
    string Text;

    TNode *ActiveNode = nullptr;
    int ActiveEdge = -1;
    int ActiveLength = 0;
    int RemainingSuffixCount = 0;
    int LeafEnd = -1;
public:
    TSuffixTree(string &text){
        this->Text += text+TERMINATION_SYMBOL;
        BuildSuffixTree();
    }
    void BuildSuffixTree(){
        ActiveNode = Root;
        for (size_t i = 0; i < Text.length(); i++) {
            ExtendSuffixTree(i);
        }
    }
    std::string Linearization(int n){
        TNode *current = Root;
        std::string res = "";
        for (int i = 0; i < n; ++i){
            auto it = current->Children.begin();
            for (;it != current->Children.end(); ++it){
                LinHelp(it->second, n, res);
                if(res != ""){
                    break;
                }
            }
        }
        return res;
    }

    ~TSuffixTree() {
        DeleteSuffixTree(Root);
    }
};
\end{lstlisting}
\begin{lstlisting}
[language=C++]
//
// main.cpp
//
#include <iostream>
#include "suffTree.hpp"
int main(){
    ios::sync_with_stdio(0);
    cout.tie(0), cin.tie(0);
    string text, pattern;
    cin >> text;
    int n = text.size();
    text = text+text;
    TSuffixTree suffixTree(text);
    std::string res = suffixTree.Linearization(n);
    std::cout << res << "\n";
    return 0;
}
\end{lstlisting}
\pagebreak

\section{Консоль}
\begin{alltt}
{\color{blue} nezhov@killswitch:~/CLionProjects/Diskran/lab5$} make
g++ -std=c++14 -O3 -Wextra -Wall -Werror -Wno-sign-compare -Wno-unused-result -pedantic -o solution main.cpp
{\color{blue} nezhov@killswitch:~/CLionProjects/Diskran/lab5$}  cat tests/01.t && ./solution < tests/01.t
ixtvoqfzlx
fzlxixtvoq
\end{alltt}
\pagebreak

