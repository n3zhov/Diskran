\section{Тест производительности}
Замеряется всё время работы программы, но, т.к. ввод одинаковый, то результаты репрезентативны. Для сравнения
производительности была выбрана библиотека GMP.
\\Количество строк (число число операция) для каждого файла равно десять в степени номер теста. 
\\Например, $02.t$ содержит сто строк, а $07.t$ десять миллионов.

\begin{alltt}
{\color{blue} nezhov@killswitch:~/CLionProjects/Diskran/lab6$} ./benchmark < tests/04.t > 04.a
custom long arithmetics 46 ms
{\color{blue} nezhov@killswitch:~/CLionProjects/Diskran/lab6$} ./benchmark1 < tests/04.t > 04.a
custom long arithmetics 55 ms
{\color{blue} nezhov@killswitch:~/CLionProjects/Diskran/lab6$} ./benchmark2 < tests/04.t > 04.a
GMP arithmetics 33 ms
{\color{blue} nezhov@killswitch:~/CLionProjects/Diskran/lab6$} ./benchmark1 < tests/06.t > 06.a
custom long arithmetics 3666 ms
{\color{blue} nezhov@killswitch:~/CLionProjects/Diskran/lab6$} ./benchmark1 < tests/05.t > 05.a
custom long arithmetics 366 ms
{\color{blue} nezhov@killswitch:~/CLionProjects/Diskran/lab6$} ./benchmark2 < tests/05.t > 05.a
GMP arithmetics 280 ms
{\color{blue} nezhov@killswitch:~/CLionProjects/Diskran/lab6$} ./benchmark1 < tests/06.t > 06.a
custom long arithmetics 3751 ms
{\color{blue} nezhov@killswitch:~/CLionProjects/Diskran/lab6$} ./benchmark2 < tests/06.t > 06.a
GMP arithmetics 3060 ms
{\color{blue} nezhov@killswitch:~/CLionProjects/Diskran/lab6$} ./benchmark1 < tests/07.t > 07.a
custom long arithmetics 40647 ms
{\color{blue} nezhov@killswitch:~/CLionProjects/Diskran/lab6$} ./benchmark2 < tests/07.t > 07.a
GMP arithmetics 30655 ms
\end{alltt}

Глядя на результаты, видно, что моя реализация однозначно уступает библиотеке GMP, но не стоит забывать,
что для большинства операций мы использовали наивный алгоритм.

\pagebreak

