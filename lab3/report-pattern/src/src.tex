\section{Дневник отладки}
1. Напишем файл benchmark.cpp, в котором замерим время работы std::map и КЧ-Дерева при помощи библиотеки <chrono> (пункт: скорость выполнения).
\\2. При помощи valgrind оценим утечки и расход памяти для std::map и собственного КЧ-Дерева (пункт: потребление оперативной памяти).
\\3. В собственной реализации не было найдено никаких утечек памяти: All heap blocks were freed -- no leaks are possible
\\4. При помощи gprof выяснили, что наибольшее время занимают операции вставки и удаления в дерево.

\pagebreak
\section{Скорость выполнения}
Тест представляет из себя следующее: при помощи генератора тестов был создан файл с $10^7$ строками, каждая из которых
является командой к дереву (добавление, удаление или поиск). Измеряем время на выполнение операций для std::map и КЧ-Дерева. 
\begin{alltt}{
    (base) nikita@nikita-desktop:~/CLionProjects/Diskran/lab2$ ./benchmark.o < tests/07.t
    ==============START============
    INSERT std::map time: 5286 ms
    INSERT rb tree time: 38894 ms
    ===============================
    DELETE std::map time: 4438 ms
    DELETE rb tree time: 19721 ms
    ===============================
    SEARCH std::map time: 1825 ms
    SEARCH rb tree time: 15717 ms
    ==============END==============
}    
\end{alltt}
Как видно, все операции из std::map выполняются быстрее, чем для собственной реализации КЧ-Дерева. Скорее всего, это связано с тем, что работу 
стандартной библиотеки C++ постоянно оптимизируют, из-за чего собственные реализации функций выполняются относительно медленнее.
\pagebreak
\section{gprof}
Скомпилируем программу с флагом -pg. Запустим нашу программу на тесте с $10^7$ входных строк. После этого в текущей директории создался файл gmoun.out,
чтобы посмотреть результат работы профилирования выполним команду gprof solution gmon.out.
\begin{alltt}{
%   cumulative   self              self     total           
time   seconds   seconds    calls  ns/call  ns/call  name    
68.05      1.74     1.74                             
NMyStd::TRBTree::Insert(NMyStd::TItem const&, NMyStd::TRBNode*)
24.64      2.37     0.63                             
NMyStd::TRBTree::Remove(char const*)
3.13      2.45     0.08  1998364    40.08    60.12  
NMyStd::TRBTree::Remove(NMyStd::TRBNode*)
1.96      2.50     0.05                             
NMyStd::TRBTree::Insert(NMyStd::TItem const&)
1.56      2.54     0.04  1161306    34.49    34.49  
NMyStd::TRBTree::RemoveFixUp(NMyStd::TRBNode*, NMyStd::TRBNode*)
0.78      2.56     0.02   853961    23.45    23.45  
NMyStd::TRBTree::InsertFixUp(NMyStd::TRBNode*)
0.00      2.56     0.00        1     0.00     0.00  
_GLOBAL__sub_I__ZN6NMyStd7TRBTree6SearchEPcRNS_5TItemE   
}    
\end{alltt}
По данной таблице можно понять, что больше всего времени уходит на вставку и удаление из дерева.
\\Дальше выводится граф вызовов (тут я вывел его в укороченном виде), в каждой строке графа указываются
функция, в предыдыущих строках выводиятся функции вызываемые данной функцией. Таким образом можно определить,
где происходят вызовы наиболее долгих функций, ведь сама функция может выполняться быстро, но функции вызываемые ей
могут тратить много времени.
\begin{alltt}{
    index % time    self  children    called     name
    <spontaneous>
[1]     68.8    1.74    0.02                 
NMyStd::TRBTree::Insert(NMyStd::TItem const&, NMyStd::TRBNode*) [1]
0.02    0.00  853961/853961      
NMyStd::TRBTree::InsertFixUp(NMyStd::TRBNode*) [6]
-----------------------------------------------
    <spontaneous>
[2]     29.3    0.63    0.12                 
NMyStd::TRBTree::Remove(char const*) [2]
0.08    0.04 1998364/1998364     
NMyStd::TRBTree::Remove(NMyStd::TRBNode*) [3]
-----------------------------------------------
0.08    0.04 1998364/1998364     
NMyStd::TRBTree::Remove(char const*) [2]
[3]      4.7    0.08    0.04 1998364         
NMyStd::TRBTree::Remove(NMyStd::TRBNode*) [3]
0.04    0.00 1161306/1161306     
NMyStd::TRBTree::RemoveFixUp(NMyStd::TRBNode*, NMyStd::TRBNode*) [5]
-----------------------------------------------
    <spontaneous>
[4]      2.0    0.05    0.00                 
NMyStd::TRBTree::Insert(NMyStd::TItem const&) [4]
-----------------------------------------------
0.04    0.00 1161306/1161306     
NMyStd::TRBTree::Remove(NMyStd::TRBNode*) [3]
[5]      1.6    0.04    0.00 1161306         
NMyStd::TRBTree::RemoveFixUp(NMyStd::TRBNode*, NMyStd::TRBNode*) [5]
-----------------------------------------------
0.02    0.00  853961/853961      
NMyStd::TRBTree::Insert(NMyStd::TItem const&, NMyStd::TRBNode*) [1]
[6]      0.8    0.02    0.00  853961         
NMyStd::TRBTree::InsertFixUp(NMyStd::TRBNode*) [6]
}
\end{alltt}    

