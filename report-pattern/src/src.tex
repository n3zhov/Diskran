\section{Описание}
Требуется написать реализацию алгоритма поразрядной сонртировки. 
Как сказано в \cite{Kormen}: \enquote{Сам алгоритм состоит в последовательной сортировке объектов какой-либо устойчивой сортировкой по каждому разряду, в порядке от младшего разряда к старшему, после чего последовательности будут расположены в требуемом порядке}. В качестве устойчивой сортировки для разряда использую устойчивую версию сортировки подсчетом\cite{wikipedia_sort}. 

Устойчивая сортировка подсчетом осуществляется с помошью двух вспомогательных массивов: один для счетчика, другой для записи отсортированного результата. Для удобства обозначу: $A$ - исходный массив, $B$ - результирующий массив, $C$ - массив для подсчета. 

Сначала заполняю $C$ нулями. Затем для каждого $A[i]$ увеличиваю $C[A[i]]$ на единицу. Суммирую каждый $C[i] с C[i - 1]$, кроме $C[0]$. Последним читаю входной массив с конца, записываю в $B[C[A[i]]]$ $A[i]$ и уменьшаю $C[A[i]]$ на единицу\cite{wikipedia_sort}.

\pagebreak

\section{Исходный код}
На каждой непустой строке входного файла располагается пара \enquote{ключ-значение}, поэтому создаю новую 
структуру $TItem$, в которой буду хранить ключ (массив из восьми $integer$) и указатель на значение (массив из 2049 элемнтов типа $char$).  Сортирую по ключу, так как копирование строки занимает слишком много времени. Так как количество пар во входных данных не указано, считываю данные в цикле $while$. Для хранения пар использую $TVector<TItem>$, который динамически выделяет память под новые пары и аналогичный TVector<char*> для хранения строк. Из ввода считываю ключ и значение как строки, затем перевожу строчный ключ в массив $integer$ и сохраняю в $TVector<TItem>$, строку сохраняю без преобразований. Так как $TVector$ перевыделяет память, если в уже выделенной не хватает места для нового элемента, то все ранее сохраненные указатели становятся невалидными. Поэтому указатели для  $TItem$ записываю после окончания ввода в отдельном цикле.

Сортирую LSD-сортировкой. Фактически за один разряд беру четырехзначное шестнадцатиричное число (массив из восьми $integer$). Иными словани перевожу 32-разрядные шестнадцатиричные числа в 8-разрядные 65 536-ричные. Делаю это для ускорения сортировки, так как ограничение по памяти позволяет. 

Выполняю сортироку в цикле, от младщего разряда к старшему. Для сортировки подсчётом выделяю массив из 65 536 элементов, обнуляю его. Подсчитываю сколько каждая пара встречалась во входных данных и записываю по соответствующим индексам. Считаю префиксные суммы, записываю в массив для посдсчёта. Теперь по индексу указано, сколько есть элементов, которвые меньше или равны элементу в индексе. Выделяю память под результирующий массив $TItem$-ов и выделяю для него память. В обратном порядке прохожусь по исходному $TVector<TItem>$ и записываю в результирующий массив элемнты, глядя на массив для подсчёта. Уменьшаю на 1 значение по индексу для текущего $TItem$. Выполняю то же самое для остальных разрядов.

Вывожу ответ.

\pagebreak
\begin{lstlisting}[language=C++]
//
// vector.hpp
//

#ifndef TVector_HPP
#define TVector_HPP
#include <cstdio>
#include <cstring>
const long long KEY_SIZE = 8;

namespace NMyStd{
    struct TItem{
        int Key[KEY_SIZE];
        char **Value;
    };

    template <class T>
    class TVector{
    private:
        long long TVectorSize;
        long long TVectorCapacity;
        T* Data;
    public:
        void Assign(const T elem);
        void Assign(const long long n, const T elem);
        TVector();
        TVector(const long long n);
        TVector(const long long n, T elem);
        long long Size();
        void PushBack(const T &elem);
        void Pop();
        bool Empty();
        T& operator[] (const long long iterator);
        ~TVector();
    };
    template <class T>
    void TVector<T>::Assign(const T elem){
        for (long long i = 0; i < TVectorSize; ++i){
            Data[i] = elem;
        }
    }

    template <class T>
    void TVector<T>::Assign(const long long n, const T elem){
        for (long long i = 0; i < n; ++i){
            Data[i] = elem;
        }
    }
    template <class T>
    TVector<T>::TVector(){
        TVectorSize = 0;
        TVectorCapacity = 0;
        Data = nullptr;
    }
    template <class T>
    TVector<T>::TVector(const long long n){
        TVectorSize = n;
        TVectorCapacity = n;
        Data = new T[TVectorCapacity];
        Assign(n, T());
    }
    template <class T>
    TVector<T>::TVector(const long long n, T elem):TVector(n){
        Assign(elem);
    }
    template <class T>
    long long TVector<T>::Size(){
        return TVectorSize;
    }
    template <class T>
    void TVector<T>::PushBack(const T &elem){
        if (TVectorCapacity == 0){
            TVectorCapacity = 1;
            TVectorSize = 0;
            Data = new T[TVectorCapacity];
        }
        else if (TVectorCapacity == TVectorSize){
            TVectorCapacity *= 2;
            T* newData = new T[TVectorCapacity];
            for (long long i = 0; i < TVectorSize; ++i){
                newData[i] = Data[i];
            }
            delete [] Data;
            Data = newData;
        }
        TVectorSize += 1;
        Data[TVectorSize - 1] = elem;
    }
    template <class T>
    void TVector<T>::Pop(){
        TVectorSize = 0;
        TVectorCapacity = 0;
        delete [] Data;
    }
    template <class T>
    bool TVector<T>::Empty(){
        return TVectorSize == 0;
    }
    template <class T>
    T& TVector<T>::operator[] (const long long iterator){
        return Data[iterator];
    }
    template <class T>
    TVector<T>::~TVector(){
        delete [] Data;
    }
}
#endif
\end{lstlisting}

\begin{lstlisting}[language=C++]
//
// main.cpp
//
#include "vector.hpp"
#include <iostream>
#include <iomanip>


const int KEY_BIT_SIZE = 4;
const int VALUE_SIZE = 2049;
const unsigned int MAX_KEY = 0xffff;
const int STR_KEY_SIZE = 33;
const int HEX_PRECISION = 4;
const char ZERO_CHAR = '0';
const int ZERO = 0;
const int ONE = 1;
const int HEX_MULTIPLY = 16;
const char A_CHAR = 'a';
const char F_CHAR = 'f';
const char NINE_CHAR = '9';

void CountingSort(NMyStd::TVector<NMyStd::TItem> &data, int bit) {
    NMyStd::TVector<long long> countArray(MAX_KEY + ONE, ZERO);
    long long size = data.Size();
    for (long long i = 0; i < size; ++i) {
        ++countArray[data[i].Key[bit]];
    }
    for (unsigned int i = 1; i <= MAX_KEY; ++i) {
        countArray[i] += countArray[i - ONE];
    }
    NMyStd::TItem *result = new NMyStd::TItem[size];
    for (long long i = size - ONE; i >= 0; --i) {
        result[countArray[data[i].Key[bit]] - ONE] = data[i];
        --countArray[data[i].Key[bit]];
    }
    for (long long i = 0; i < size; ++i) {
        data[i] = result[i];
    }
    delete [] result;
}

void BitwiseSort(NMyStd::TVector<NMyStd::TItem> &data) {
    for (int i = KEY_SIZE - ONE; i >= 0; --i) {
        CountingSort(data, i);
    }
}

int main() {
    std::cin.tie(nullptr);
    std::cout.tie(nullptr);
    std::ios_base::sync_with_stdio(false);
    NMyStd::TVector<NMyStd::TItem> data;
    char strKey [STR_KEY_SIZE];
    NMyStd::TItem cur;
    cur.Value = nullptr;

    char bufInput [VALUE_SIZE];
    NMyStd::TVector<char*> valueData;

    while (std::cin >> strKey >> bufInput) {
        for (int & i : cur.Key) {
            i = ZERO;
        }
        for (int i = KEY_SIZE - ONE; i >= 0; --i) {
            int hexMultiply = ONE;
            for (int j = KEY_BIT_SIZE - ONE; j >= 0; --j) {
                if (strKey[i * KEY_BIT_SIZE + j] >= ZERO_CHAR && strKey[i * KEY_BIT_SIZE + j] <= NINE_CHAR) {
                    cur.Key[i] += (strKey[i * KEY_BIT_SIZE + j] - ZERO_CHAR) * hexMultiply;
                }
                else if (strKey[i * KEY_BIT_SIZE + j] >= A_CHAR && strKey[i * KEY_BIT_SIZE + j] <= F_CHAR) {
                    cur.Key[i] += (strKey[i * KEY_BIT_SIZE + j] - A_CHAR + 10) * hexMultiply;
                }
                hexMultiply *= HEX_MULTIPLY;
            }
        }
        char *curValue = new char[VALUE_SIZE];
        std::memcpy(curValue, bufInput, sizeof(char)*VALUE_SIZE);
        data.PushBack(cur);
        valueData.PushBack(curValue);
    }
    for (int i = 0; i < valueData.Size(); ++i) {
        data[i].Value = &valueData[i];
    }
    BitwiseSort(data);
    for (int i = 0; i < data.Size(); ++i) {
        for (int j = 0; j < KEY_SIZE; ++j) {
            std::cout << std::hex << std::setw(HEX_PRECISION) << std::setfill(ZERO_CHAR) << data[i].Key[j];
        }
        std::cout << " " << *data[i].Value << "\n";
    }
    for (int i = 0; i < valueData.Size(); ++i) {
        delete[] valueData[i];
    }
    return 0;
}
\end{lstlisting}

\begin{longtable}{|p{7.5cm}|p{7.5cm}|}
\hline
\rowcolor{lightgray}
\multicolumn{2}{|c|} {vector.hpp}\\
\hline
TVector()&Конструктор TVector, обнуляет размер, емкость и указатель на данные.\\
\hline
\hline
TVector(const long long n, T elem);&Конструктор TVector, задаёт вектору размер и ёмкость n, и забивает данные эл-ом elem.\\
\hline
Size()&Метод TVector, возвращает размер.\\
\hline
PushBack(const T \&elem)&Метод TVector, добавляет новый элемент шаблонного класса T в конец TVector.\\
\hline
operator[](const long long iterator)&Перегруженный оператор []. Возвращает данные по итераторру.\\
\hline
\hline
Assign(const long long n, const T elem)&Функция, которая задаёт вектору размер и ёмкость n, и забивает данные эл-ом elem\\
\hline
\end{longtable}
\pagebreak

\section{Консоль}
\begin{alltt}
{\color{blue} (base) nikita@nikita-desktop:~/Diskran/lab1$ make}
g++ -O3 -std=c++14 -o solution main.cpp vector.cpp vector.h
{\color{blue} (base) nikita@nikita-desktop:~/Diskran/lab1$ cat test_input.txt}
a5db3d43b37a95cd00618a7ac3112f7e LQFZzcjKUIgaNHdxMhDRzSojQdKKdJRxqntO
d40c0348390bdb0e0fee9348cc8d2e67 vVNAzruNkZUPUUqHcmfWgkwdOGTSJeaAINZ
364590e5c89b6b1c54231793d261a3b2 wMmpTNRfxIrkciktSkSdLYabpVgehC
734bf391f564bd5aff79a1fefeb358b7 m
e31adffb1b4418881731733600387d82 vJCqcbdkeEVJLWwRbPTRWk
caf836a9f342e48deb43e8215b23b177 FbGyVbPqdyvlQoJoqmvdiaCboIkNOILfldOAPtjxbzoVUxFKSOvppxRCtf
5a3378d5ca2706bffab672e07894212b QKutCZfccVqEZORVvVxPICHQYAOemh
0472d854d83d11c287ec7d9eff9b8146 CvqgXrzHlEWFRIORDYhvJH
c41d2af1b376b9fc1f95fce356012712 SYhTAeNdhtZlTAWOreQeMNPaGgFSYNRMpoldNrUDsEAIR
8a18a352c07e3af6411007385bffd0ed ibdyiGQrGFQbLZngBwpsNoMiUrIGvkQwHwYGwVSWZ
{\color{blue} (base) nikita@nikita-desktop:~/Diskran/lab1$ ./solution < test_input.txt}
0472d854d83d11c287ec7d9eff9b8146 CvqgXrzHlEWFRIORDYhvJH
364590e5c89b6b1c54231793d261a3b2 wMmpTNRfxIrkciktSkSdLYabpVgehC
5a3378d5ca2706bffab672e07894212b QKutCZfccVqEZORVvVxPICHQYAOemh
734bf391f564bd5aff79a1fefeb358b7 m
8a18a352c07e3af6411007385bffd0ed ibdyiGQrGFQbLZngBwpsNoMiUrIGvkQwHwYGwVSWZ
a5db3d43b37a95cd00618a7ac3112f7e LQFZzcjKUIgaNHdxMhDRzSojQdKKdJRxqntO
c41d2af1b376b9fc1f95fce356012712 SYhTAeNdhtZlTAWOreQeMNPaGgFSYNRMpoldNrUDsEAIR
caf836a9f342e48deb43e8215b23b177 FbGyVbPqdyvlQoJoqmvdiaCboIkNOILfldOAPtjxbzoVUxFKSOvppxRCtf
d40c0348390bdb0e0fee9348cc8d2e67 vVNAzruNkZUPUUqHcmfWgkwdOGTSJeaAINZ
e31adffb1b4418881731733600387d82 vJCqcbdkeEVJLWwRbPTRWk
\end{alltt}
\pagebreak

