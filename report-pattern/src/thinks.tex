\section{Выводы}

Выполнив первую лабораторную работу по курсу \enquote{Дискретный анализ}, я научился реализовывать шаблонный класс с динамическим выделением памяти на примере реализации класса $TVector$. Реализовал поразрядную сортировку и стабильную сортировку подсчетом. Узнал, что оператор копирования для массива $char$-ов работает очень медленно, соответсвенно выгодно сортировать по указателям. Хоть на больших тестах($10^5$ и больше) соритровка по разрядам быстрее чем $std::stable\_sort$, но пространсвенная сложность $O(n + m)$ (где $n$ - рамер входного массива, а $m$ - размер разряда) дает о себе знать. Также из-за сложности $O(n*m)$ поразрядная сортировка очень медленно работает для чисел, где много разрядов, из-за чего она может быть медленее $std::stable\_sort$, поэтому конкретно в этом варианте задачи целесообразнее разделить 32-ричное число на 8 разрядов по 4 числа
\pagebreak
