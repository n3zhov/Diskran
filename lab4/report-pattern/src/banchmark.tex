\section{Тест производительности}
Реализованный алгоритм Кнута-Морриса-Пратта с использнованием прфикс функции сравнивается
с наивным алгоритмом поиска. Замеры производятся на тестах из $10^3$ слов, $10^4$ или $10^5$. 
Длина образца в первом тесте - 10, во втором - 25, в третьем - 100

\begin{alltt}
nezhov@LAPTOP-VDTFKP8R:/mnt/c/Users/nikit/Diskran/lab2$ ./benchmark < tests/03.t
==============START============
Naive: 0.822 ms
KMP: 0.049 ms
==============END==============
nezhov@LAPTOP-VDTFKP8R:/mnt/c/Users/nikit/Diskran/lab2$ ./benchmark < tests/04.t
==============START============
Naive: 5.418 ms
KMP: 0.564 ms
==============END==============
nezhov@LAPTOP-VDTFKP8R:/mnt/c/Users/nikit/Diskran/lab2$ ./benchmark < tests/05.t
==============START============
Naive: 25.715 ms
KMP: 2.412 ms
==============END==============
\end{alltt}

Видно, что наивный алгоритм поиска почти на порядок проигрывает алгоритму Кнута-Морриса-Пратта. 
Классический алгоритм допускает лишние сравнения на этапе поиска образца в тексте, а алгоритм с применение префикс функции - нет.
Обработка таких сравнений длиться дольше, чем предпроцессинг префикс функции.
\pagebreak

