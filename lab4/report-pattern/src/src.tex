\section{Описание}
Алгоритм Кнута-Морисса-Прутта 
\\Требуется реализовать алгоритм Кнута-Морриса-Пратта для поиска 
подстроки в строке. Учитывая, что алфавит состоит из регистронезависимых слов не более
16 знаков, нужно уметь правильно представлять переводы строки, пробелы и табуляции.
\\ Согласно \cite{Gusfield}, алгоритм Кнута-Морриса-Пратта прикладывает образец к тексту и начинает делать сравнение
с левого конца. В случае полного совпадения, было найдено вхождение, слвигаем образец на один символ вправо. Если же есть
несовпадения, то мы делаем сдвиг по особому правилу, в отличии от алгоритма наивного поиска, который всегда сдвигает на 
один символ.
\\Для каждой позиции i определим $sp_i(P)$ как длину наибольшего собственного суффикса $P[1..i]$, который совпадает с префиксом
$P$, причём символы в позициях $i+1$ и $sp_i(P) + 1$ различны.
\\$sp$ - префикс функция. Опишем эффективный алгоритм её построения. Но для начала рассмотрим наивный алгоритм. Наивный алгоритм вычисляет префикс-функцию непосредственно по определению, сравнивая префиксы и суффиксы строк. 
Обозначим длину строки за $n$. Будем считать, что префикс-функция хранится в массиве $p$. Как сказано в \cite{Algo}, такой алгоритм имеет сложность $O(n^2)$. Теперь внесём в алгоритм пару замечаний.
\\Вносятся несколько важных изменений:
\\Заметим, что $p[i + 1] \leq p[i] + 1$. Чтобы показать это, рассмотрим суффикс, оканчивающийся на позиции $i + 1$ и имеющий длину $p[i + 1]$, удалив из него послдний символ, мы получим суффикс, оканчивающийся на позиции $i$
и имеющий длину $p[i + 1] - 1$, следовательно неравенство $p[i + 1] > 1$ неверно.
\\Избавимся от явных сравнений строк. Пусть мы вычислили $p[i]$, тогда, если $s[i + 1] = s[p[i]]$, то $p[i + 1] = p[i] + 1$. Если окажется, что $s[i+1] \neq s[p[i]]$, то нужно попытаться попробовать подстроку меньшей длины.
Хотелось бы сразу перейти к такому бордеру наибольшей длины, для этого подберем такое $k$, что $k=p[i] - 1$. Делаем это следующим образом. За исходное $k$ необходимо взять $p[i - 1]$, что следует из первого пункта. В случае,
когда символы $s[k]$ и $s[i]$ не совпадают, $p[k-1]$ - следующее потенциальное наибольшее значение $k$. Последнее утверждение верно, пока $k > 0$, что позволит всегда найти его следующее значение. Если $k = 0$, то $p[i] = 1$ при
$s[i] = s[1]$, иначе $p[i] = 0$.

\pagebreak

\section{Исходный код}
В файле kmp.hpp содержутся функции для КМП, а также структура TWord, которая используется для хранения слова, его номера в строке, и номер строки, в которой он находится.
\begin{center}Листинг kmp.hpp\end{center} 
\begin{lstlisting}[language=C++]
//
// kmp.hpp
//

#ifndef LAB4_KMP_HPP
#define LAB4_KMP_HPP
#include <vector>
#include <string>
namespace NSearch {
    struct TWord{
        std::string Word;
        long long WordId, StringId;
    };
    std::string tolower(std::string &data){
        int size = data.size();
        for (int i = 0; i < size; ++i){
            data[i] = std::tolower(data[i]);
        }
        return data;
    }
    std::vector<long long> PrefixFunction(const std::vector<TWord> &input) {
        std::vector<long long> res(input.size());
        long long length = input.size();
        for (long long i = 1; i < length; ++i) {
            long long j = res[i - 1];
            while(j > 0 && input[i].Word != input[j].Word){
               j = res[j - 1];
            }
            if (input[i].Word == input[j].Word){
                ++j;
            }
            res[i] = j;
        }
        return res;
    }
    std::vector<long long> KMP(std::vector<TWord>&pattern, std::vector<TWord> &text, std::vector<long long> &prefix){
        long long patternSize = pattern.size();
        long long textSize = text.size();
        long long i = 0;
        std::vector<long long> answer;
        if (patternSize > textSize || patternSize == 0) {
            return answer;
        }
        while (i < textSize - patternSize + 1) {
            long long j = 0;
            while (j < patternSize && pattern[j].Word == text[i + j].Word) {
                ++j;
            }
            if (j == patternSize) {
                answer.push_back(i);
            }
            else{
                if (j > 0 && j > prefix[j - 1]) {
                    i = i + j - prefix[j - 1];
                    continue;
                }
            }
            ++i;
        }
        return answer;
    }
}
#endif //LAB4_KMP_HPP
\end{lstlisting}
В файле \textit{main.cpp} будем посимвольно считывать текст. После считывания запускаем функцию поиска КМП, после чего выводим полученные ответы.
\begin{center}Листинг main.cpp\end{center}
\begin{lstlisting}[language=C++]
//
// main.cpp
//
#include <iostream>
#include "kmp.hpp"
int main(){
    bool flagPattern = true;
    std::cin.tie(nullptr);
    std::cout.tie(nullptr);
    std::ios_base::sync_with_stdio(false);
    std::vector<NSearch::TWord> pattern;
    std::vector<NSearch::TWord> text;
    NSearch::TWord input;
    std::string word;
    char c = getchar();
    while (c > 0) {
        if (c == '\n') {
            if (!input.Word.empty()) {
                text.push_back(input);
                input.Word.clear();
            }
            ++input.StringId;
            if (flagPattern) {
                std::swap(pattern, text);
                flagPattern = false;
                input.StringId = 1;
            }
            input.WordId = 1;
        } else if (c == '\t' || c == ' ') {
            if (!input.Word.empty()) {
                text.push_back(input);
                input.Word.clear();
                ++input.WordId;
            }
        } else {
            if ('A' <= c and c <= 'Z') {
                c = c + 'a' - 'A';
            }
            input.Word += c;
        }
        c = getchar();
    }
    if (!input.Word.empty()) {
        text.push_back(input);
    }
    std::vector<long long> prefix = NSearch::PrefixFunction(pattern);
    std::vector<long long> ans = KMP(pattern, text, prefix);
    for (long long & id : ans) {
        std::cout << text[id].StringId << ", " << text[id].WordId << '\n';
    }
    return 0;
}
\end{lstlisting}
\pagebreak

\section{Консоль}
\begin{alltt}
nezhov@LAPTOP-VDTFKP8R:/mnt/c/Users/nikit/Diskran/lab4\$ cat tests/01.t
cat dog cat dog bird
CAT dog CaT Dog Cat DOG bird CAT
dog cat dog bird
nezhov@LAPTOP-VDTFKP8R:/mnt/c/Users/nikit/Diskran/lab4\$ cat tests/01.a
1, 3
1, 8
nezhov@LAPTOP-VDTFKP8R:/mnt/c/Users/nikit/Diskran/lab4\$ make
g++ -std=c++17 -O3 -Wextra -Wall -Werror -Wno-sign-compare -Wno-unused-result -pedantic -o solution main.cpp
nezhov@LAPTOP-VDTFKP8R:/mnt/c/Users/nikit/Diskran/lab4\$ ./solution < tests/01.t > res.txt
nezhov@LAPTOP-VDTFKP8R:/mnt/c/Users/nikit/Diskran/lab4\$ diff res.txt tests/01.a
\end{alltt}
\pagebreak

